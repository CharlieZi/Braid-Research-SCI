
\begin{abstract}
This paper develops a model for a specific type of hose construction designed to withstand very high operating pressures. The model is based on a model previously developed by Entwistle and White  with two significant modifications. Firstly the compressible inner core is included in the model using Lame’s thick walled cylinder theory. Secondly the model allows for the squeezing effect on wires when a hose gets shorter under pressurization. The model calculates whether the wires in a particular layer will be squeezed together and when this occurs, the behavior is modeled using Hertzian contact theory. The governing equations are solved using a minimizing Newton Raphson technique. Model predictions are compared with experimental results obtained for pressure deformation response in terms of hose axial strain and wire strain and show good agreement. Considerable hysterical behavior is seen in the hose axial strain and it is suggested that this may be due to the twisting contact movements between different layers and as such may be a good indicator of the amount of fretting taking place. It is also suggested that, when a hose is designed to get shorter on pressurization, length change may be a good indicator of manufacturing quality in terms of wire packing efficiency. 
\end{abstract}




