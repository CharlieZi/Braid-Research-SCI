\begin{keyword}
%% keywords here, in the form: keyword \sep keyword
braid \sep
PTFE \sep 
hose
%% MSC codes here, in the form: \MSC code \sep code
%% or \MSC[2008] code \sep code (2000 is the default)
\end{keyword}