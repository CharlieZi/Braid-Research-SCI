\section{Introduction}
\label{introduction}




This paper is concerned on the high pressure flexible hoses reinforced by braided metallic wires, which are utilized in a variety of engineering applications to transmit fluid in the aerospace, automobile, marine and aviation industries[]. The braid reinforced flexible hoses are practically employed in more severe hydraulic conditions where high pressure loads are not static but periodically or randomly fluctuating, and furthermore thermal loading and large deformation are coupled with the pressure[], commonly of the order of tens of MPa. 
The construction of such a hose is illustrated in Figure 1. It comprises an inner PTFE tube core with four layers of high tensile steel wires wounding around it, such that the PTFE resists leakage and chemical corrosion while the steel wires layers comprise the principle load-carrying elements. The reinforcement layer consists of two helix-wound layers and outer two braid layers (see Figure 1 (b))
The helix-wound layers are wounded in pairs, one layer of each pair being wound left hand and the other right hand in order to achieve a torque balanced construction (i.e. minimal twist on pressurization)[]. There are no intermediate layers of plastic and wires in the same layer are touching in order for maximum packing density.
Braiding is formation of rope-like structure by diagonally interlacing several units of wires, called spindles (usually between four and eight wires depending on the diameter of the tube []). 
In conventional braider, spindle carriers rotate along a circular track. Half of the carriers travel in clockwise direction, with the others in the reverse direction, similar to maypole arrangement (see Figure 1).As a result, he two sets of spindles interlace with each other at a bias angle to the tube axis, namely the braid angle, which play a pivot part in defining the performance of the hose under pressure.[1]
The tube in Figure 1 is designed a hybrid structure with both the helix-wound reinforcement and the braid reinforcement, to have sufficient structural safety against extremely high pressure. They may be independently used in middle-high pressure hose. Braided fiber yarns, especially Kevlar [], has been used instead of metallic wires, for its light weight. Effects have been made to model and characterize the mechanical and structural elements of the three reinforcement methods respectively.

just have fun



